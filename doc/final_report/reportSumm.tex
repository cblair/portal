
%%%%%%%%% SUMMARY -- 1 page, third person
% e.g:  "The PI will prove" not "I will prove"

%Introduction / Summary (1 page).  The following information is typically included in proposal introductions for this 
%grant. The summary should lead to the proposal body but it should also be a self-contained document, 
%summarizing what is being proposed and why.   
 
% Provide appropriate background that sets the context for the problem and the objectives. 
%State the specific problem (may also be thought of as a need, central research question, or hypothesis) 
%your research will address. 
% State the objectives of the proposed research. 
%State any limits of the proposed research  (may not be needed). 
%Summarize the research methods you will use to achieve research objectives (if clearly covered elsewhere 
%in the introduction, a separate section may not be needed). 


%\required{Project Summary}
\section{Project Summary}
% This should be a brief statement of the problem you plan to address.
% It should look something like an abstract. 

\subsection{Background}

In the areas of ecology and biology, there is an explosion of data. The
amount of bioinformatics data in research institutions doubles every 9 months 
\cite{james}. Gains in processing power, however, are decreasing, as the end
of Moore's Law's approaches \cite{gordon_moore}. As processor speed gains decrease, 
the gap between data and results is growing. More and more data has 
to be ignored. Data that was expensive to collect.

In spite of the growth of data and platue of processor computing power, solutions 
exists that allows scientists and mathematicians to conduct their 
research. High Performance Computing (HPC) is one approach, but is not yet widely
available. Most researchers have to instead be more efficient with what data they 
run computations on, and what data they ignore.

\subsection{Problem Statement}

Currently, researchers' data lie in lots of flat files, in a directory somewhere
on a computer. If they are lucky, then their data actually lives in a customized
database somewhere. But for those who are lucky, the database is hard to use, and 
is designed specifically for their data. If a new format for the data is seen, 
much of the core database code has to be rewritten. This creates a lag and a cost
between researchers and their tools.

In both cases, accessing and sharing research data is also expensive. The files 
structures are too big and hard to filter, and the database interfaces are not
very user friendly. The result is a lot of research organizations, with duplicate
data. They also have lots of data that they want to share with each other, and 
would mutually benefit each other. But they have no scalable solutions for 
sharing the data.

\subsection{Objectives}

The objectives in the Hatch tool are to research existing database technologies
and concepts, and find a solution for the data deluge. It needs to have an 
extremely simple interface, and reuse use cases that researchers already
know. Other objectives are to create useful tools that help researchers
filter and vizualize their data, and take the pain out of data management.
The objective is not to finish the Hatch tool, but to provide a good 
code and concept foundation for further development. 

%\required{Intellectual Merit}
% This is why your project is interesting and will help further
% knowledge in the field of mathematics. 

%\required{Broader Impacts}
% There are 4 kinds of broader impacts.
% 1. advance discovery and understanding while promoting teaching,
% training and learning
% 2. broaden the participation of underrepresented groups
% 3. disseminated broadly to enhance scientific and technological
% understanding
% 4. benefits of the proposed activity to society

