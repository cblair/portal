
%%%%%%%%% MASTER -- compiles the sections

\documentclass[11pt,letterpaper]{article}
\setlength{\parskip}{1ex plus 0.5ex minus 0.2ex}

\usepackage{graphicx}
\usepackage{hyperref}
\hypersetup{
    colorlinks,%
    citecolor=black,%
    filecolor=black,%
    linkcolor=black,%
    urlcolor=black
}
\usepackage{listings} %for source code figures
\usepackage[toc,xindy,style=long3colheaderborder,footnote]{glossaries}
%\usepackage[toc,xindy]{glossaries}
\makeindex
\makeglossaries


\newglossaryentry{speedup}{name={speedup}, description={The overall time gain of a process or computation.}}

\newglossaryentry{time complexity}{name={time complexity}, description={The time it takes for a process to run.}}


\newglossaryentry{paralellization}{name={paralellization}, description={Taking independent or semi-independent processes, and processing them in parallel (at the same time).}}

\newglossaryentry{granularity}{name={granularity}, description={The time or portion of the process that each iteration / sub-process takes.}}

\newglossaryentry{HPC}{name={HPC}, description={High Performance Computing.}}


%%%%%%%%%%%%%%%%%%%%%%%%%%%%%%%%%%%%%%%%%%%%%%%%%%%%%%%%%%%%%%%%%%%%%%%%%
\pagestyle{plain}                                                      %%
%%%%%%%%%% EXACT 1in MARGINS %%%%%%%                                   %%
\setlength{\textwidth}{6.5in}     %%                                   %%
\setlength{\oddsidemargin}{0in}   %% (It is recommended that you       %%
\setlength{\evensidemargin}{0in}  %%  not change these parameters,     %%
\setlength{\textheight}{8.5in}    %%  at the risk of having your       %%
\setlength{\topmargin}{0in}       %%  proposal dismissed on the basis  %%
\setlength{\headheight}{0in}      %%  of incorrect formatting!!!)      %%
\setlength{\headsep}{0in}         %%                                   %%
\setlength{\footskip}{.5in}       %%                                   %%
%%%%%%%%%%%%%%%%%%%%%%%%%%%%%%%%%%%%                                   %%
\newcommand{\required}[1]{\section*{\hfil #1\hfil}}                    %%
\renewcommand{\refname}{\hfil References Cited\hfil}                   %%
\bibliographystyle{plain}                                              %%
%%%%%%%%%%%%%%%%%%%%%%%%%%%%%%%%%%%%%%%%%%%%%%%%%%%%%%%%%%%%%%%%%%%%%%%%%

%PUT YOUR MACROS HERE

\date{Spring 2012}
\title{Team EcoData - Hatch Tool\\
}

\author{Mike Solomon\\
	Colby Blair \\
	Computer Science Undergraduates \\
	University of Idaho Computer Science Department\\
	CS 481 Capstone Project \\
}


\begin{document}

\section*{Letter of Transmittal}
\textbf{Subject.} High Performance Computing in Natural Resources.

\textbf{Purpose.} This report focuses on Hatch project, and the challenges with managing data in the
scientific research environment. The goal is to introduce new approaches to organizing data, and making
it more searchable. The Hatch project creates a foundation for a tool that allows users to upload, search,
format, and visualize their data in ways they didn't think of before. This report highlights some of the 
problems with computation and data management in the current research environment, and makes 
suggestions for new approaches. It talks about specific algorithms for searching data, and standards for
formatting and storing it.

\textbf{Background.} The amount of data being stored in scientific databases today is growing 
exponentially. The amount of computational power in cpu's, however, is no longer growing exponentially.
The consequence is an gap opening between data and results. Researchers are trying to build and rebuild
their computational infrastructure, to try to get computational growth back to the pace of data growth.
But in this report, we suggest a complimentary approach to computing research data. The Hatch project
comes from the idea that a significant amount of research data is redundant or not wanted. But it is also
vast and not managable. Researchers end up wasting computation time on data points and sets that they do 
not care about, at the expense of ones they do care about.

\textbf{Preliminary Work}. With the experience of the authors of this report, there is more than 8 years of work in computer science, bioinformatics, and natural resources. There is 2+ years of work in computing
clusters for the University of Idaho Initiative for Bioinformatics and Evolutionary STudies (IBEST), as well as 
1.5 years of research in wildlife management. 

\thispagestyle{empty}

\pagebreak

\maketitle

\thispagestyle{empty}

\pagebreak

\thispagestyle{empty}
\tableofcontents
\listoffigures

\pagebreak

\begin{abstract}
With today's exponential increase of data, the demand for processing is outpacing the supply \cite{gordon_moore}. 
The way we 
processed data yesterday would take years to do today. Unfortunately, the lack of fluency with computing in
today's research proposals results in processing a fraction of the data collected. This weakens research 
projects, and leads to a lot of unnecessary data collection. 

Today's researchers need tools to become more efficient with processing data. Until computation can keep
up again with exponential data growth, researchers will need to carefully select the data that they want to 
do research on, and throw away data that will not yield results. The users of such tools range anywhere from
fish researchers in the Columbia Basin, to bioinformaticians, to economists, and more. Having more 
searchable, usable data applies to almost everyone, and the use cases and usefulness of a tool like Hatch
are endless.

\end{abstract}

\setcounter{page}{1}

\pagebreak

%
%%%%%%%%% SUMMARY -- 1 page, third person
% e.g:  "The PI will prove" not "I will prove"

%Introduction / Summary (1 page).  The following information is typically included in proposal introductions for this 
%grant. The summary should lead to the proposal body but it should also be a self-contained document, 
%summarizing what is being proposed and why.   
 
% Provide appropriate background that sets the context for the problem and the objectives. 
%State the specific problem (may also be thought of as a need, central research question, or hypothesis) 
%your research will address. 
% State the objectives of the proposed research. 
%State any limits of the proposed research  (may not be needed). 
%Summarize the research methods you will use to achieve research objectives (if clearly covered elsewhere 
%in the introduction, a separate section may not be needed). 


\section{Background}
%this section needs more stats and citations
In the Columbia Basin today, millions of federal dollars are spent on PTAGIS
and other systems to collect environmental data.  This data includes fish
location, ecological community composition, and abiotic data. Yet, the project
results from most research are far from concrete. Despite the lack of
understandable results, important decisions that affect the local environment
and economy have to be made. Decisions like whether or not to conduct major
habitat restoration projects are sometimes made without convincing data to
suport them. 

%We should cite something other study / examples in the basin. Foster's talk
% just sticks in my head as a good example
The Columbia Basin is just one small example. Bioinformatics is another 
data intensive field that generates more data that it can process. It is 
%is this 10%? It may be lower
estimated that less than 10\% of the data collected by bioinformatic 
researchers at the University of Idaho actually makes it through processing
\cite{foster}. The rest has to be filtered as best as can be managed, and the 
low value data trimmed out. The problem with 10\% data use is that not just 
the fat, but the meat and bone has to be cut away. Either significantly more
data must be analyzed, or significantly less should be collected. At the very
least, storing the data in an easily readable format can show where the 
gaps are. So far, our experience and research shows similar shortfalls in 
data analysis in the Columbia Basin.

There is a clear need for a tool that simplifies data management.  Specifically,
a tool is needed that allows users to work remotely with data, perform basic manipulations
and filters, visualize data points, and explore data.  We expect that such a
tool would not only simplify the scientific process for researchers with large
data sets, but allow the processing of a larger percentage of relevant data.

\section{Problem Definition}
One of the biggest problems researchers in the Columbia Basin have is moving
their data somewhere meaningful. Central databases like PTAGIS offer a 
central storage and clients can push data there, but they don't offer useful
tools for managing data. Once the data is pushed, it is hard to access and
manipulate. 
Researchers are often in remote locations, and have low bandwidth
connections, but would benefit greatly from a centralized location with 
built-in data management and exploration capabilities.
Creating a robust tool will guarantee ease of use for the data, no matter
the location.

Many researchers have significant data management tasks before even thinking 
about pushing data to the cloud. Once they are ready, they need seamless ways to
synchronize data to and from the cloud. They also need to query their data,
and filter it into small subsets. Most researchers don't have time to learn
new programming languages or interfaces. They need a simple data management tool 
that has an intuitive user 
interface and fits their use cases. Once they
have created a data subset, they will want to share it, save it, copy it, 
and compare it with other data sets. Getting the right data to the right place 
in the right amount of time is crucial.

In this interest, a data management tool needs to be written.  It must meet the
following requirements:

\singlespacing
\begin{itemize}
    \item Provide:
    \begin{itemize}
        \item Reliable data storage
         \item Basic data manipulations
         \item Simple visualizations
     \end{itemize}
     \item Have a gentle learning curve
     \item Allow remote access
     \item Allow local access
     \item Support many data sets (i.e.,  not just PTAGIS data)
 \end{itemize}

 \doublespacing

\section{Objectives}

\begin{figure}[!h]
        \begin{center}
		\includegraphics[width=120mm]{images/combo_proposal}
                \caption{This proposal software's application domain in DE-CRRP}
                \label{combo_proposal}
        \end{center}
\end{figure}

The proposed project will implement the UI data harvester portion of the grant
funded USGS-CRRP project (Figure~\ref{combo_proposal}), directed by Alex
Fremier of the UI College of Natural Resources. The data management tool will
use a web-based GUI that can be installed locally on the clients' computers,
but may also be accessed remotely.  The tool will contain a expandable meta
data server that can be connected to others to form a replicated data storage
system. It will use an existing RESTful networking protocol that will allow for many
different data formats to be synchronized across the cloud. The project will
also have a graphical, web-based interface for simplified data management.

The replicated in-cloud model allows some significant advantages to
researchers.  They can see exactly what their data looks like before they
submit it, due to the integrated visualizations.  They can filter out bad data
before it consumes bandwidth, and can retract undesirable data from the cloud,
even after synchronizing it. It also frees them from central storage service
management and fees, and encourages internode and inter-research communication.
The tool can be set up on multiple hosts, and can be used for the benefits of
the centralized cloud model. Each client can decide what topology suites them
best.

This model uses a distributed database model.
Distributed databases increase availability and reliability through automatic
replication. They are more
easily expandable, and can better protect from data loss from local disasters or
malicious attacks. Moving data to where it is in highest demand also increases
query performance. Offloading archive data to remote site with more 
resources preserves local resources. Replicated datasets can guarantee 
better availability. By staging data locally and filtering it before allowing
it to be exchanged, the autonomy of the organization is better preserved, and the
relevance of uploaded data can be improved. 

%probably some specific network protocols, like SOAP, etc, should be mentioned
%Alex, the network protocol doesn't have to apply just to low bandwidth users
%, but also to large data transfers
The network protocol will allow incremental synchronization of data from host
to host, even in less reliable environments. The tool will create
an outreach from researchers in high availability areas to those
in low availability, low bandwidth areas, and back. The network protocol
will have support for major data formats such as CSV, and allow
users to send incremental pieces of the data.

The intuitive interface will allow users to sort and manipulate their data,
needing only basic knowledge of computers (and naturally, the data in
question). Simple queries will create data subsets that users can bring to
a workspace. The tool will be able to sort data however the user likes, on the
fly. It will then be able to graph the ranges in the subset in most ways the
user could want to sort them.

Once the user has manipulated their data subset to their satisfaction, they
will be able to save instantly on the server and download it as a CSV file. The
tool will also be extensible so that future analysis modules will allow them to
run analysis on the local host, or remotely. These modules will be capable of
running analysis jobs on other designated compute hosts, like workstations,
clusters, or even Amazon's EC2. The tool will come only with basic
functionality for local analysis modules, but will be extensible for future, heavier
compute options.
%\required{Intellectual Merit} This is why your project is interesting and will
%help further knowledge in the field of mathematics. 

%\required{Broader Impacts}
% There are 4 kinds of broader impacts.
% 1. advance discovery and understanding while promoting teaching,
% training and learning
% 2. broaden the participation of underrepresented groups
% 3. disseminated broadly to enhance scientific and technological
% understanding
% 4. benefits of the proposed activity to society

 % - this stuff is ok, but I think the proposal stuff is better

%%%%%%%%% SUMMARY -- 1 page, third person
% e.g:  "The PI will prove" not "I will prove"

%Introduction / Summary (1 page).  The following information is typically included in proposal introductions for this 
%grant. The summary should lead to the proposal body but it should also be a self-contained document, 
%summarizing what is being proposed and why.   
 
% Provide appropriate background that sets the context for the problem and the objectives. 
%State the specific problem (may also be thought of as a need, central research question, or hypothesis) 
%your research will address. 
% State the objectives of the proposed research. 
%State any limits of the proposed research  (may not be needed). 
%Summarize the research methods you will use to achieve research objectives (if clearly covered elsewhere 
%in the introduction, a separate section may not be needed). 


\section{Background}
%this section needs more stats and citations
In the Columbia Basin today, millions of federal dollars are spent on PTAGIS
and other systems to collect environmental data.  This data includes fish
location, ecological community composition, and abiotic data. Yet, the project
results from most research are far from concrete. Despite the lack of
understandable results, important decisions that affect the local environment
and economy have to be made. Decisions like whether or not to conduct major
habitat restoration projects are sometimes made without convincing data to
suport them. 

%We should cite something other study / examples in the basin. Foster's talk
% just sticks in my head as a good example
The Columbia Basin is just one small example. Bioinformatics is another 
data intensive field that generates more data that it can process. It is 
%is this 10%? It may be lower
estimated that less than 10\% of the data collected by bioinformatic 
researchers at the University of Idaho actually makes it through processing
\cite{foster}. The rest has to be filtered as best as can be managed, and the 
low value data trimmed out. The problem with 10\% data use is that not just 
the fat, but the meat and bone has to be cut away. Either significantly more
data must be analyzed, or significantly less should be collected. At the very
least, storing the data in an easily readable format can show where the 
gaps are. So far, our experience and research shows similar shortfalls in 
data analysis in the Columbia Basin.

There is a clear need for a tool that simplifies data management.  Specifically,
a tool is needed that allows users to work remotely with data, perform basic manipulations
and filters, visualize data points, and explore data.  We expect that such a
tool would not only simplify the scientific process for researchers with large
data sets, but allow the processing of a larger percentage of relevant data.

\section{Problem Definition}
One of the biggest problems researchers in the Columbia Basin have is moving
their data somewhere meaningful. Central databases like PTAGIS offer a 
central storage and clients can push data there, but they don't offer useful
tools for managing data. Once the data is pushed, it is hard to access and
manipulate. 
Researchers are often in remote locations, and have low bandwidth
connections, but would benefit greatly from a centralized location with 
built-in data management and exploration capabilities.
Creating a robust tool will guarantee ease of use for the data, no matter
the location.

Many researchers have significant data management tasks before even thinking 
about pushing data to the cloud. Once they are ready, they need seamless ways to
synchronize data to and from the cloud. They also need to query their data,
and filter it into small subsets. Most researchers don't have time to learn
new programming languages or interfaces. They need a simple data management tool 
that has an intuitive user 
interface and fits their use cases. Once they
have created a data subset, they will want to share it, save it, copy it, 
and compare it with other data sets. Getting the right data to the right place 
in the right amount of time is crucial.

In this interest, a data management tool needs to be written.  It must meet the
following requirements:

\singlespacing
\begin{itemize}
    \item Provide:
    \begin{itemize}
        \item Reliable data storage
         \item Basic data manipulations
         \item Simple visualizations
     \end{itemize}
     \item Have a gentle learning curve
     \item Allow remote access
     \item Allow local access
     \item Support many data sets (i.e.,  not just PTAGIS data)
 \end{itemize}

 \doublespacing

\section{Objectives}

\begin{figure}[!h]
        \begin{center}
		\includegraphics[width=120mm]{images/combo_proposal}
                \caption{This proposal software's application domain in DE-CRRP}
                \label{combo_proposal}
        \end{center}
\end{figure}

The proposed project will implement the UI data harvester portion of the grant
funded USGS-CRRP project (Figure~\ref{combo_proposal}), directed by Alex
Fremier of the UI College of Natural Resources. The data management tool will
use a web-based GUI that can be installed locally on the clients' computers,
but may also be accessed remotely.  The tool will contain a expandable meta
data server that can be connected to others to form a replicated data storage
system. It will use an existing RESTful networking protocol that will allow for many
different data formats to be synchronized across the cloud. The project will
also have a graphical, web-based interface for simplified data management.

The replicated in-cloud model allows some significant advantages to
researchers.  They can see exactly what their data looks like before they
submit it, due to the integrated visualizations.  They can filter out bad data
before it consumes bandwidth, and can retract undesirable data from the cloud,
even after synchronizing it. It also frees them from central storage service
management and fees, and encourages internode and inter-research communication.
The tool can be set up on multiple hosts, and can be used for the benefits of
the centralized cloud model. Each client can decide what topology suites them
best.

This model uses a distributed database model.
Distributed databases increase availability and reliability through automatic
replication. They are more
easily expandable, and can better protect from data loss from local disasters or
malicious attacks. Moving data to where it is in highest demand also increases
query performance. Offloading archive data to remote site with more 
resources preserves local resources. Replicated datasets can guarantee 
better availability. By staging data locally and filtering it before allowing
it to be exchanged, the autonomy of the organization is better preserved, and the
relevance of uploaded data can be improved. 

%probably some specific network protocols, like SOAP, etc, should be mentioned
%Alex, the network protocol doesn't have to apply just to low bandwidth users
%, but also to large data transfers
The network protocol will allow incremental synchronization of data from host
to host, even in less reliable environments. The tool will create
an outreach from researchers in high availability areas to those
in low availability, low bandwidth areas, and back. The network protocol
will have support for major data formats such as CSV, and allow
users to send incremental pieces of the data.

The intuitive interface will allow users to sort and manipulate their data,
needing only basic knowledge of computers (and naturally, the data in
question). Simple queries will create data subsets that users can bring to
a workspace. The tool will be able to sort data however the user likes, on the
fly. It will then be able to graph the ranges in the subset in most ways the
user could want to sort them.

Once the user has manipulated their data subset to their satisfaction, they
will be able to save instantly on the server and download it as a CSV file. The
tool will also be extensible so that future analysis modules will allow them to
run analysis on the local host, or remotely. These modules will be capable of
running analysis jobs on other designated compute hosts, like workstations,
clusters, or even Amazon's EC2. The tool will come only with basic
functionality for local analysis modules, but will be extensible for future, heavier
compute options.
%\required{Intellectual Merit} This is why your project is interesting and will
%help further knowledge in the field of mathematics. 

%\required{Broader Impacts}
% There are 4 kinds of broader impacts.
% 1. advance discovery and understanding while promoting teaching,
% training and learning
% 2. broaden the participation of underrepresented groups
% 3. disseminated broadly to enhance scientific and technological
% understanding
% 4. benefits of the proposed activity to society


%\include{reportSoftware.text} %should include what technologies we use; rails, couchdb, etc
\subsection{Database}

\subsubsection{Introduction}
One of the biggest challenges with Hatch was how to organize data. Specifically,
most organizations with scientific data have their own standard or format on 
how they store research data. Many of these organizations want to share data between 
each other, but they have a hard time reaching agreement on how to merge the formats.
Consider the following examples:

\begin{figure}[h]
	\begin{center}
	\begin{tabular}{ | c | c | c | }
		\hline
		site	&	datetime		&	unique fish tag	\\
		\hline
		TUC	&	02/16/06 19:08:15 	&	3D9.1BF1E7919A 	\\
		TUC	&	02/16/06 19:18:36 	&	3D9.1BF1A998FA 	\\
		TUC	&	02/17/06 18:21:03 	&	3D9.1BF20E8FE2	\\
		...	&	...			&	...		\\
		\hline
	\end{tabular}
	\caption{A database representing for PTAGIS data} 
	\label{ptagis_ex1}
	\end{center}
\end{figure}

\begin{figure}[h]
	\begin{center}
	\begin{tabular}{ | c | c | }
		\hline
		unique fish tag	&	DNA sequence	\\
		\hline
		3D9.1BF1E7919A 	&	ATGCTTAC...	\\
		3D9.1BF1A998FA 	&	TTACGATC...	\\
		3D9.1BF20E8FE2	&	GTGGASCT...	\\
		...		&	...		\\
		\hline
	\end{tabular}
	\caption{A database representation for DNA data} 
	\label{dna_ex1}
	\end{center}
\end{figure}

In each of the examples above, the data is represented with \textbf{rows} and
\textbf{columns}, much the same way someone would represent the data in a
spreadsheet, such as Microsoft Excel. These structures in a typical
relational database (MySQL, etc).  are called tables.

The above examples are simplifications of the rows and columns in actual research 
data, but they highlight one of the biggest issues with data storage using relational
databases: they require you to know the column names ahead of time. Not only that,
but they require that you know the data types of the values that go in those columns,
and once a table is created expecting a certain format, it is hard to change.

The problem with needing to know the structure of research data before designing
databases is that research data is semi-structured at best. Once it does
represent some structure, it often changes. For example, once researchers finally 
decide what columns and data types should go in the table in Figure~\ref{ptagis_ex1},
another researcher may suggest more columns that should go in to the table.

This leads to endless edits to the database and program design by some software 
developer. The standard table format that everyone can agree on isn't useful to many
researchers, because it usually leaves out many other needed columns and fields,
or includes many irrelevant fields.

A better approach is needed. Researchers, not committees, should decide how to store 
data. Data should be mergable based on common values in different tables (like the 
unique fish tag column in Figures~\ref{ptagis_ex1} and~\ref{dna_ex1}. The 
person who enters the data should decide how one particular dataset is stored in a 
database, and should be able to choose to store the same data in a different table 
format as they choose. There should be a simple tool that helps them do this.

The following sections describe different approaches to implementing a database design
that enables data storage for dynamic or semi-structured data.


\subsubsection{Relation Databases: per-document table creations}
This approach is the simplest and follows the concept of table creation for data sets
pretty closely. Basically, for each input document in the form of a spreadsheet, a new
SQL table is created. The columns names and type are determined from the headers and 
data values in the spreadsheet.

\begin{figure}[h]
	\begin{center}
	\begin{lstlisting}
		CREATE TABLE ptagis_doc1
			(
				id int, 
				site char(50), 
				read_data_time date,
				tag char(50)
			); 
	\end{lstlisting}
	\caption{The SQL syntax for creating the table in Figure~\ref{ptagis_ex1} } 
	\label{ptagis_ex1_sql}
	\end{center}
\end{figure}

The biggest problem with this approach is that each document
in the database is a table. When searching for a specific document, the database 
typically searches for the table name. This search is linear, and with hundreds,
thousands, or hundreds of thousands of documents, frequently searching the database
to look for values would be increasingly slow and therefore useless.

Another problem with this design is that building software to support this would
be difficult and complicated, since it is not regarded as a good practice.


\subsubsection{Relational Databases: tables for each datatype}
Another approach is to create column tables for each data type, and let document 
tables just be collections of columns. Each of the document column values point to 
respective values in the column tables.

\begin{figure}[h]
	\begin{center}
	\includegraphics[width=80mm]{images/rel_db_lookup}
	\caption{Document table as a lookup table} 
	\label{rel_db_lookup}
	\end{center}
\end{figure}

This allows for documents to have a dynamic number of columns with variable data types,
but there are two problems with this approach. First, every value of a given type in every document
in the database is put into one table (e.g.\ all values with a `string' data type
go into the `string' table). With potential millions of data values, each table
becomes an overflowing bucket and doesn't utilize the advantages of storing multiple
columns and values in one table. Searches for data would require lots of filtering 
for just the data required from specific doucments and would therefore be inefficient.

The other issue is that every retrieval of data from a document would require
many lookups. Data retrieval over significantly large data sets would quickly become
very computationally intensive, and eventually  impractical.


\subsubsection{CouchDB}
When one thinks about the fundamental issues with storing, searching, and merging 
research data, a core issue is identified: data is semi-structured. This is what
makes trying to use relational databases so hard. They were made for datasets where
you knew the structure up front, and seldom wanted to change their structures.

The one assumption that Hatch makes about the data is that \textbf{there are rows and
columns}. This is the only assumption Hatch makes. This leaves the database and 
interface designs free from whatever changes are needed by the users.

This is done by storing data in JSON format. The data is this lightweight format,
modeled exactly like Ruby on Rails 3 returns records from its Active Record.
This allows users to input data however they like, define delimiters for columns using
Hatch Input Filters, and Hatch does the rest. It finds the most specific data type the
values can be stored in, skips non-matching entries, and populates all the forms on 
the web pages according to the data, all without dictating the structure of the 
data, or even knowing it before hand.

This leads to the technical implementation of the Hatch Database.


\subsubsection{Hatch Database}
Since Hatch uses Ruby on Rails, there are lots of tools and libraries for using the 
standard relational database, via Rails' Active Record. In many/most cases, Hatch
actually wants to stick to the relational database. With Hatch's internal database
structure, order is important. For example, a user will always have a name, email
address, etc. A document will always have a name and an owner. However, the data in the
document is the semi-structured data, and Hatch only wants to use CouchDB for that for
the reasons described above.

\begin{figure}[h]
	\begin{center}
	\includegraphics[width=120mm]{images/hatch_db_hybrid}
	\caption{The Hatch relational - non-relational database hybrid} 
	\label{hatch_db_hybrid}
	\end{center}
\end{figure}

The result is a relational--non-relational database hybrid. Hatch uses traditional 
relational database columns when the columns are fixed and known in advance (as is
often the case for information such as user names and e-mails), and can add columns to
a database entity (called a scaffold) on the fly using CouchDB. For example, we create
a scaffold called Documents. Document always have a name, id, and collection/folder
they belong to. But, documents may have a ``data'' section, or they may not. That
data section may have arbitrary numbers of columns and rows of data that would match the flat 
file they came from (like an Excel file). The document may instead have any other data that
can be stored in JSON format. It is up to the user of the Hatch interface, not the database, to 
decide. By allowing Hatch to infer the structure and fields of data, Hatch
is by extension allowing the user to decide how to format data.

Most Rails applications that use CouchDB completely replace Active Record with some
library's version of it, like couchrest's Active Model. However, by replacing
Active Record, you lose support for libraries that lots of Rails developers
make, like pagination package \texttt{will\_paginate}. 

Hatch gets around this by using its database hybrid through a Ruby library called
stuffing. Stuffing is a link that ties Rails Active Record records with
CouchDB documents. It allows for rapid prototyping in development, and is a nice, modest
alternative to replacing the internals of Active Record. Since the base model for 
database scaffolds is still Active Record, the huge amount of Rails Active Record based
libraries still work.

\pagebreak

\begin{figure}[h]
	\begin{center}
	\includegraphics[width=120mm]{images/couchdb_json_ex}
	\caption{Typical representation of document data in CouchDB / JSON} 
	\label{couchdb_json_ex}
	\end{center}
\end{figure}

\subsection{Search}
After the EcoData team addressed the issues with storing semi-structured data, the next 
big problem to address was how to efficiently search through the data. The interface 
that was desired was one much like Google's search; a simple search box, with two
buttons. The interface needs to be extremely simple yet powerful for it to be 
effective for non-technical users. 

\begin{figure}[h]
	\begin{center}
	\includegraphics[width=120mm]{images/search_ss1}
	\caption{The search interface} 
	\label{search_ss1}
	\end{center}
\end{figure}

For example, when a user entered in a search like `3D9.1', it would be assumed that 
any data starting with `3D9.1' should be returned. Because Hatch assumes that
the user would want any data in the same row as the matching data, it returns
the entire row. 

For search, this could lead to a huge number of string comparisons in order to find
all values that match. This would make search impractically slow. Luckily, CouchDB allows
us to implement a practical solution to this problem.

\subsubsection{Views}
CouchDB uses precompiled queries called views. Views take developer-defined
query templates and apply them to every document that is created or updated in the 
database when the document is saved. The results are precompiled lookup tables 
(actually heaps/binary trees), which make searches fast. 

Hatch basically creates a view like the following pseudocode:
\singlespacing
\begin{lstlisting}
	for each document
		for each row
			for each column value in row
				emit(value, row)
\end{lstlisting}
\doublespacing

\texttt{emit()} is a function that tells CouchDB how to create the search B-Tree. It 
takes two arguments; the key that the tree node will take, and the value that the 
node will return if the key matches the search. Hatch says `every column value in a 
document is a key, and the return value is the data row it belongs to,' so if a search
matches a document value, the search results return the entire data row.

CouchDB makes Hatch's job easier by having internal methods for string matching. For
example, if a search for `3D9.1' is used with CouchDB startkey, CouchDB will return
any string starting with `3D9.1'. Hatch doesn't have to invent a query language to 
tell the database lots of parameters for matching values, which means users do not
need to learn a new query language either.

\begin{figure}[h]
	\begin{center}
	\includegraphics[width=120mm]{images/couchdb_b_tree}
	\caption{CoucbDB search through its internal binary tree.} 
	\label{couchdb_b_tree}
	\end{center}
\end{figure}

The other advantage to CouchDB is that it stores values in the B-Tree structure. For
string data types, this means that the database searches only within the `3D9.1' string range
in the tree. When the search sees a child node starting with `3D9.0', it instead picks
the child node staring with `3D9.1', and the entire `3D9.0' group is never searched 
through, which makes searches complete in a short time frame. 

\subsection{Visualization}

When users are searching their data, it is important to get rapid feedback on the meaning,
shape, and size of the data they have found.  An intuitive method for this is through
charts, graphs, and other visualizations.

Hatch's visualizations allow users to accomplish these tasks. They allow users
to see unique data and graph values in comparison with other values. The Hatch
Documents interface allow users to do things like categorize data and then use
visualizations to graph the results.

\begin{figure}[h]
	\begin{center}
	\includegraphics[width=120mm]{images/viz_ex1}
	\caption{A visualization example.} 
	\label{viz_ex1}
	\end{center}
\end{figure}

Hatch uses HighCharts, a JavaScript library, to perform visualizations. It
allows Hatch to incrementally stream new graph data to the client, instead of
resending all the graph data. This is useful when users are working on low
bandwidth connections, or with large datasets. By reducing unneeded data 
transfer, Hatch is able to work with larger datasets faster.

Hatch is able to check incrementally for new data. When data is graphed, it points towards the 
data in the document that needs visualization. If the data in the document changes,
the graph adds the data points to the graph without refreshing the page.

This opens up the possibility of Feeds. Feeds are scheduled events in Hatch that pull
JSON data from somewhere on the web. They push the data in to the document they point
to. This data can simultaneously be used by a visualization in order to immediately graph
the new data, as soon as a change is detected in the document.
The result is Live Charts, and documents that sync with data on the web when new data is 
available. Instead of doing a bulky request for data all at once, data can be taken
in smaller requests, and Hatch can give users a real-time view their data as it comes
into the system


\section{Conclusion}
In conclusion, the Hatch project had great aspirations, and fell short. But in these
aspirations, design patterns were followed to allow futher development to continue
along a sound path. Despite not getting everything done the EcoData team set out to 
do, we did accomplish some really great things with the tool. We were able to show
the client some important ideas and concepts that they didn't realize the gravity of
until the ideas were implemented.

Hatch defines a new model for aggregating, storing, and searching data. It embraces
RESTful design and the DRY mantra. Hatch decouples design from data at all times;
something that is not too common in scientific research. As a result, the features
that are added to Hatch will continue to support any kind of data, as long as that 
data follows the core assumption Hatch makes; data has rows and columns.

Hatch embraces JSON as much as possible; Hatch's Rails views all have json version of
the web pages. Feeds syncs with JSON data on the web, and stores data in CouchDB's
native JSON format. The result is embracing open data exchange through the web.
Although the decentralized database syncing is not yet implemented, the core for it
is there, and the future functionality of Hatch is only limited by developers and 
ideas.


%%%%%%%%% REFERENCES -- no limit

% this should include only items referenced in the project description
% it is not a bibliography of related reading.

% Each reference must include the names of all authors (in the same
% sequence in which they appear in the publication), the article and 
% journal title, book title, volume number, page numbers, and year of 
% publication. If the document is available electronically, the website 
% address also should be identified

\section{Bibliography}

\begin{thebibliography}{99}
\bibitem{james} Foster, James. {\em Visualizing Human Microbiome
        Ecosystems}. University of Idaho: Computer Science Colloquium, 
        December 7th 2010. Seminar.	
\bibitem{gordon_moore} Manek Dubash (2005-04-13). "Moore's Law is dead, says 
	Gordon Moore". {\em Techworld}. Retrieved 2006-06-24
\bibitem{amdahl} Amdahl, Gene (1967). "Validity of the Single Processor Approach to Achieving Large-Scale Computing Capabilities". {\em AFIPS Conference Proceedings} (30): 483–485. Print.
\bibitem{rocks} Copyright (c) 2000 - 2010 The Regents of the University of California. All rights reserved.
\bibitem{bb} Horne, Garton, et al. "Analyzing Animal Movements using Brownian 
	Bridges". { \em Ecology } 88(9) 2007: 2354-2363. Print
\bibitem{syn} Horne, Garton, Rachlow. "A synoptic model of animal space use: 
	Simultaneous estimation of home range, habitat selection, and 
	inter/intra-specific relationships" {\em Ecological Modelling} 
	214, 2008: 338-348. Print.

%\bibitem{new_life} Kanellos, Michael (19 April 2005). {\em New Life for 
%Moore's Law}. cnet. Retrieved 2009-03-19

\end{thebibliography}

%\printglossaries



\end{document}
