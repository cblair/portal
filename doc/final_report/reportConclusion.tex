
\section{Conclusion}
In summary, this project showed even the author some suprising aspects of parallelism. In our own project,
the tail end of Amdahl's Law came much sooner than we thought. Applying parallel concepts can be difficult,
even for professionals. But it doesn't need to be. The concepts are simple in nature, and if good coding 
standards are accepted from the beginning, then the translation into parallel code will be easy.

The speedups that are seen in parallelism are very important. Without them, science of tomorrow will not be
possible. Researchers need to become familiar with the limitations of even parallel processing, and adjust their
project proposals accordingly. Not doing so will lead to failure. And a lack of understanding of parallelism may
cause a project to fail unnecessarily. HPC is the language of today, and we must all speak it to answer any 
questions about tomorrow.