\section{Future Work}

Hatch has been specifically designed to be extensible and modular.  Because the needs
of the customer are so wide-ranging and would require a larger effort than our small
development team could reasonably complete in this time frame, Hatch has been
constructed with future extension in mind since day one.

Several major areas exist which could readily be improved or extended.  Included are
a few of these areas along with brief descriptions and time estimates, based on a team of
two working about 15 hours per week each.  Many other parts of Hatch could be extended,
but these are some of the most important areas for scientific research.

\subsection{New visualization types}
Currently, Hatch only fully supports single-series spline, line, and scatterplot
charts. Most of the code exists to extend this to categorical types and the
corresponding charts, such as pie and bar charts.  In addition, supporting
multiple-series data on a single chart is another obvious way to improve Hatch.

\textbf{Time estimate:} 1 week per chart type; 2 weeks for multiple-series data

\subsection{New data manipulation methods}
Hatch supports basic data manipulations such as filtering, sorting, and bringing
together columns from different data sets.  Obvious and useful extensions include
single-step joins between different data sets, log transforms, summary columns,
and suggested data joins.  These would form the basis for a more sophisticated
data manipulation pipeline that could be separated into a distinct package,
if desired.

\textbf{Time estimate:} 2 weeks per manipulation

\subsection{Performance enhancements}
Some areas of Hatch are not as responsive as could be.  While performance is not
crippling to the software, it can occasionally take a few seconds to complete
an action such as a search.  Some of this can be solved through more powerful
hardware (most tests were run on aging laptops), but algorithmic improvements
as well as general optimization could improve responsiveness greatly.

\textbf{Time estimate:} 1 month to double performance (more if other features are added)

\subsection{Cross-instance replication and sharing}
The original vision for Hatch included a system where each researcher could
run a localized Hatch instance and use the fully functional software on their
own machine.  Later, this data could be shared with a centralized Hatch instance,
or shared with other researcher's Hatch instances.  This requires solutions for
several difficult but solvable problems, and would require extensive testing.

\textbf{Time estimate:} 5 months

\section{Conclusion}
The Hatch project attempted to solve several difficult problems in an easy-to-use
package.  Due to time and resource constraints, Hatch must be seen as the beginning
of a comprehensive data management system that Hatch is positioned to grow into over time
through future efforts.

The developers followed accepted design patterns and consciously constructed the
software to be easily extended in every area.  Since it is likely this software
will be developed further in the future, a considerable amount of effort was spent
ensuring that each piece could be replaced or extended as necessary. Although the
project does not meet every need the customer has, it accomplished the goal of 
either meeting crucial needs or putting most of the pieces in place to allow the
straightforward extension of Hatch to meet them in the future.

Hatch defines a new model for aggregating, storing, and searching data. It embraces
RESTful design and the Don't Repeat Yourself mantra.
Hatch decouples design from data at all times;
something that is not too common in scientific research. As a result, the features
that are added to Hatch will continue to support any kind of data, as long as that 
data follows the core assumption Hatch makes: that data is in rows and columns.
Hatch embraces JSON as much as possible, providing JSON views whenever possible.
Feeds sync with JSON data on the web, and stores data in CouchDB's
native JSON format. The result is embracing open data exchange through the web in
one of the most ubiquitous and universal formats.

Although the certain features are not yet implemented, the essential core has
been demonstrated to work as desired, and the future functionality of Hatch is
only limited by developers and ideas.
