
\section{Conclusion}
In conclusion, the Hatch project had great aspirations, and fell short. But in these
aspirations, design patterns were followed to allow futher development to continue
along a sound path. Despite not getting everything done the EcoData team set out to 
do, we did accomplish some really great things with the tool. We were able to show
the client some important ideas and concepts that they didn't realize the gravity of
until the ideas were implemented.

Hatch defines a new model for aggregating, storing, and searching data. It embraces
RESTful design and the DRY mantra. Hatch decouples design from data at all times;
something that is not too common in scientific research. As a result, the features
that are added to Hatch will continue to support any kind of data, as long as that 
data follows the core assumption Hatch makes; data has rows and columns.

Hatch embraces JSON as much as possible; Hatch's Rails views all have json version of
the web pages. Feeds syncs with JSON data on the web, and stores data in CouchDB's
native JSON format. The result is embracing open data exchange through the web.
Although the decentralized database syncing is not yet implemented, the core for it
is there, and the future functionality of Hatch is only limited by developers and 
ideas.
