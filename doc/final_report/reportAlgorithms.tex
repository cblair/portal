
\section{Algorithms}
This report consider algorithms used in CNR, because much of the time, they calculate values over 
environments. The data then is often in vector or matrix form. Both of these data types get used a lot in 
iteration, which is why the language R is useful. It is also why parallelism is useful.

\subsection{Brownian Bridge}
The Brownian Bridge algorithm for analyzing animal movements has many areas for potential parallelization.
The most advantagious is the estimation:

\begin{figure}[h!]
        \begin{center}
                $h(z) = \frac{1}{T_{total}} \sum_{i=0}^{n-1} \int_0^{T_i} \! \delta(z; \mu(t),\sigma_i^2(t)) \mathrm{d}t.$
                \caption{Brownian Bridge estimation \cite{bb}}
                \label{bb_est}
        \end{center}
\end{figure}

Exercising parallel concepts, the granularity has the potential to be big. And indeed it is. For this equation,
$i$ animal locations are considered (conceptually a vector). Running this algorithm in R on 979 animal locations,
 $P \approx 38 $ (seconds), $(1 - P) \approx 2.5 $ (seconds) from Figure \ref{amdahl}. Running on our cluster, 
our $T_P \approx 0.00892$ . This is an example of an algorithm that is embarassingly parallel. Running these
values through Amdahl's Law in Figure \ref{amdahl}, one can find the optimal number of CPUs to run 
the algorithm:

\begin{figure}[h!]
        \begin{center}
                %\includegraphics[width=80mm]{images/opt_cpus_979.png}
                \caption{Optimal CPUs (65) for the Brownian Bridge estimation with 979 animal locations}
                \label{opt_cpus_979}
        \end{center}
\end{figure}

\pagebreak

For 3486 animal locations, $P \approx 105$, $(1 - P) \approx 108$, with the same $T_P \approx 0.00892$.
The result is:

\begin{figure}[h!]
        \begin{center}
                %\includegraphics[width=80mm]{images/opt_cpus_3486.png}
                \caption{Optimal CPUs (108) for the Brownian Bridge estimation with 3486 animal locations}
                \label{opt_cpus_3486}
        \end{center}
\end{figure}

\subsection{Synoptic Model}
Another great candidate is the Synoptic Model of animal space use, using Brownian Bridge variance \cite{syn}.
The logic algorithm is not only embarrasingly parrellel, it is rediculously parallel. The logic function is as
follows:

\begin{figure}[h!]
        \begin{center}
                $L(\theta, \beta) = \sum_{q=0}^{n} ln[\frac{ f_0(x_q|\theta) \prod_{i=1}^k (1 + \beta_i H_i(x_q)) }{  \int_x [f_0(x_q|\theta) \prod_{i=1}^k (1 + \beta_i H_i(x))] }]$
                \caption{Synoptic Model Likelihood algorithm \cite{syn}}
                \label{bb_est}
        \end{center}
\end{figure}

For 979 animal locations, $P \approx 4000$, $(1 - P) \approx 116$, with the same $T_P \approx 0.00892$.
The result is:

\begin{figure}[h!]
        \begin{center}
                %\includegraphics[width=80mm]{images/opt_cpus_979_syn.png}
                \caption{Optimal CPUs (670) for the Synoptic Model \cite{syn} with 979 animal locations}
                \label{opt_cpus_syn_979}
        \end{center}
\end{figure}


%\subsection{Hidden Markov Models}
%Hidden Markov Models for DNA Sequencing
