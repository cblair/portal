\subsection{Data I/O}
\label{sec:data_io_spec} 

\subsubsection{Input}
	\begin{enumerate}
		\item The input module will allow users to import CSV native 
			formats into the metadata server’s native format.
		\begin{enumerate}
			\item \textit{Verify that the CSV inputs correctly
					into Metadata as a new Data instance.}
		\end{enumerate}
		
		\item The input module will allow creation of new input filters
			as regular expressions, to be applied to every line
			of a file, for custom file format input/importing. This
			will be implemented by the Filter scaffold (see 
			Section \ref{sec:filter}).
		\begin{enumerate}
			\item \textit{Verify that a user can specify filters
				and then apply them to correctly parse and 
				import files, per the expected behavior
				of regular expressions.}
		\end{enumerate}

		\item Once a file is inputed, the wall and system time will be
			displayed, along with the number of rows created.
		\begin{enumerate}
			\item \textit{ Verify that the wall and system time, 
					along with the number of rows, are
					displayed on input.}
		\end{enumerate}

		\item When a file is inputed, a Metadata instance will be 
			created or updated, with the 'name' field set during 
			the import.
		\begin{enumerate}
			\item \textit{ Verify that the Metadata instance is
					created with the 'name' set as the
					file name. }
		\end{enumerate}

		\item When a file is inputed, it is represented as a Data 
			instance with the 'name' field set as the name of the
			imported file.
		\begin{enumerate}
			\item \textit{ Verify that the inputed file creates 
					a new Data instance, and each row's
					name column is set to the name of the
					inputed file.}
		\end{enumerate}

		\item When a file is inputed, each column will be intepreted
			as a Rails 3 data type (see Section
			\ref{sec:data_types}). The column data type will be 
			applied to all members of all rows for the respective
			columns.
		\begin{enumerate}
			\item \textit{ Verify that a column is intepreted to
					be a support Data Type, and is inputed.}
		\end{enumerate}

		\item When a column is being considered for which Data Type
			it is, it will be attempted to be interpreted as a 
			data type in the following order, until success:
			\begin{enumerate}
				\item time
				\item date
				\item timestamp
				\item datetime
				\item boolean
				\item binary
				\item decimal
				\item float
				\item integer
				\item string
				\item text
			\end{enumerate}

		\begin{enumerate}
			\item \textit{ Verify each kind of data is intepreted
					correctly.}
		\end{enumerate}

		%\item
		%\begin{enumerate}
		%	\item \textit{ Verify }
		%\end{enumerate}
	\end{enumerate}


\subsubsection{Data Types}
\label{sec:data_types}

The supported data types are (in list of precedence):
\begin{enumerate}
	\item time
	\item date
	\item timestamp
	\item datetime
	\item boolean
	\item binary
	\item decimal
	\item float
	\item integer
	\item string
	\item text
\end{enumerate}
