\subsection{Documents - Requirements}
\label{sec:document_spec} 

\subsubsection{Scaffold}
The following are the requirements for Document scaffolds:
\begin{enumerate}
	\item Active Record
	\begin{enumerate}
		\item Documents shall have a \textbf{name} field.
		\item Documents shall have a \textbf{collection} field, which is an ID 
			of the parent Collection.
		\item Documents shall have a belongs\_to relationship with Collections,
			Users, and Feeds.
		\item Documents shall have a has\_many relationship with Charts.
		\item Documents shall have at least all the scaffold file generated by
			\textbf{rails generate scaffold attributes}.
	\end{enumerate}

	\item CouchDB
	\begin{enumerate}
		\item The CouchDB document will be named 'Document-<ID>', where the
			ID field is the corresponding Document ID in Active Record.
		\item All documents will have a \textbf{data} member. If empty, they
			will be an empty JSON array ( [] ).i
		\item The 'data' member is an array of hashes, and should match the
			structure that Rails Active Record has for records.
	\end{enumerate}
\end{enumerate}

\subsubsection{Scaffold - Views}
The following are requirements for Document views:
\begin{enumerate}
	\item edit - view shall be the standard rails generated view. 

	\item index - view shall start from the standard rails generated view. 
	\begin{enumerate}
		\item View shall show a Search form at the top.i
		\item If this document is a result of a search, it shall display
			a link to 'Show as Document' for search results, and then
			display the data member search results below.
		\item If a search has been submitted, then show a list of Documents
			whose names also match the search.
		\item For each Document listing displayed, display an additional link
			that allows for Document CSV download.
	\end{enumerate}

	\item edit - view shall be the standard rails generated view. 

	\item show - view shall start from the standard rails generated view. 
	Additionally:
	\begin{enumerate}
		\item View shall display the \_visualize partial.
		\item View shall display a form for data manipulation.
		\begin{enumerate}
			\item Form will have a radio button for each column that can
				manipulated.
			\item Form will have a selection list of manipulation functions.
				The only current function is \textbf{categorize}.
		\end{enumerate}
		\item View shall display all the data from the CouchDB document, as a
			html table.
		\begin{enumerate}
			\item Each cell in the data table shall be a link to Document
				Hatch Lucky Search.
		\end{enumerate}
	\end{enumerate}

\end{enumerate}


\subsubsection{Helpers}
The following are the requirements for Document Helpers:
\begin{enumerate}
	\item get\_data\_colnames(d)
	\begin{enumerate}
		\item param d - A CouchDB 'data' member of a Document. 
		\item Return - a list of all 'column names' (unique keys) from the 
			structure.
	\end{enumerate}

	\item convert\_data\_to\_native\_types(d)
	\begin{enumerate}
		\item param d - A CouchDB 'data' member of a Document. 
		\item Return - A CouchDB 'data' member of a Document, with data types
			changed from string to more explicit type, if possible.
		\item The type conversion priority is listed as follows:
		\begin{enumerate}
			\item datatime
			\item decimal/float
			\item integer
			\item string
		\end{enumerate}
	\end{enumerate}

	\item get\_data\_column(d, colname)
	\begin{enumerate}
		\item param d - A CouchDB 'data' member of a Document. 
		\item param colname - A string for the colname wanted.
		\item Return - A CouchDB 'data' member of a Document, but with only
			hashes with key == colname.
	\end{enumerate}

	\item get\_data\_map(d, colname)
	\begin{enumerate}
		\item param d - A CouchDB 'data' member of a Document. 
		\item param colname - A string for the colname wanted.
		\item Return - A CouchDB 'data' member of a Document, with unique
			values for column == colname in '1' column, and the count of
			each of the values in '2' column.
	\end{enumerate}

	\item document\_search\_data\_couch(search, lucky\_search = false)
	\begin{enumerate}
		\item param search - A string for the search value
		\item param lucky\_search - A bool value, defaults to false
		\item Return -  A CouchDB 'data' member of a Document:
		\begin{enumerate}
			\item If lucky\_search == false, a CouchDB startkey search.
			\item Else, a CouchDB key search.
		\end{enumerate}
	\end{enumerate}

\end{enumerate}
